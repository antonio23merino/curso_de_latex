\documentclass[a5paper, 12pt]{article}
%%%%%%%%%%%%%%%%%%%%%%
\usepackage[utf8]{inputenc}
\usepackage[spanish]{babel}
\usepackage[top=2cm, left=2cm]{geometry}
\usepackage{amsmath, amssymb, amsfonts, latexsym}
\usepackage{graphicx}
\usepackage{color}
\usepackage{multicol}
\usepackage{nicefrac}
%%%%%%%%%%%% para poner un yingyang
\usepackage{ marvosym }
%para separar mas las columnas
%%%%%%%%%%%%

\setlength\columnsep{10pt}
%%%%%%%%%%%%%
%para quitar sangría
\parindent = 0mm
%%%%%%%%%%%%%%%%%%%%%
\begin{document}


Sea \(x\) un número real positivo y sea \(y \mathbb(Z)\) tales que
	\(
	x + yr = 0,
	\) 
	para todo \(r \in \mathbb(R)\). Demostraremos que \(x)\) es \(0\)
	
\vspace{2\baselineskip}

%%%%%%%%%%%%%%
% con parentesis se genera una sola linea, con corchetes da un espacio para la formula matematica	
Sea \(x\) un número real positivo y sea \(y \mathbb(Z)\) tales que
	\[
	x + yr = 0,
	\] 
	para todo \(r \in \mathbb(R)\). Demostraremos que \(x)\) es \(0\). Para ello notemos que si
	\[
	x + yr = 0,
	\],
	para todo \(r \in \mathbb(R)\), entonces podemos tomar
	\[
	r:=-\dfrac{x}{y}
	\]
	
	para todo \(r \in \mathbb(R)\). Demostraremos que \(x)\) es \(0\). Para ello notemos que si
	\[
	x + yr = 0,
	\]	
	
%%%%%%%%%%%%%%%%

\vspace{2\baselineskip}
\begin{multicols}{2}


Sea \(v \in M\) un vector de norma diferente de 0, buscamos todos los vectores tales que su producto punto con \(v\) sea 1; este no es el ortogonal de \(v\). es más, dicho conjunto no es un espacio vectorial, es en efecto un afín o un hiperplano. Este hiperplano es de la forma
	\[
	H=\{u \in M :\ v \cdot u = 1\} . + yr = 0,
	\]	

\end{multicols}

%%%%%%
%para buscar simbolos raros http://detexify.kirelabs.org/classify.html
%%%%

Ahora vamos a resolver la ecuación diferencial:

\[
	-\Delta + 3 = u^2
\]

\[
	\alpha
\]

%%%%%%%%%%%%%%%%%%%%%
%%%%%%%%%%%%%%%%%%%%%

Para todo \(\varepsilon	> 0 \) existe \(\delta >0 \) tal que
\[
	|x-a|< \delta \Rightarrow |f(a) - f(x) | < \varepsilon.
\]

%%%%%%%%%%%%%%%% en linea
\vspace{2\baselineskip}
asdasdasd
\(
	\dfrac{\mathbb R \times \mathbb Z \times \mathbb Q }{2}
\)
asdasdasd
\vspace{2\baselineskip}

%%%%%%%%%%%%%%%% en bloque
asdasdasd
\[
	\mathbb {R} \times \mathbb {Z} \times \mathbb {Q}/2
\]
asdasdasd

\begin{equation}
	\dfrac{a}{b} \geq 0 \quad \Leftrightarrow \quad (a\geq 0 \wedge b \geq 0) \lor (a\leq 0 \wedge b \leq 0).
\end{equation}

	
Pero una solución es \(0\), por lo tanto la EDP tiene infinitas soluciones. Es más, podríamos intentar con una función \(v\) y evaluar la siguiente expresión en l EDP:
\[
	\dfrac{v^2}{v} .	
\]	

\[
	a_{16}^{ja}
\]

\[
	\lim_{x \rightarrow 5} f(a)
\]

\[
	\lim_{x \to 5} f(a)
\]

Esto no es lo mismo que \(nicefrac{v}{2} \) o \( \dfrac{v}{2}\).

\vspace{2\baselineskip}

\(
	\displaystyle \lim_{x \to 5} f(a)
\)	

\vspace{2\baselineskip}	

Consideremos la sucesión \((x_{n})_{n \in \mathbb{N}} \) de término general
\[
	x_{n}= \dfrac{(-1)^n}{n+1}.
\]	
	
\vspace{2\baselineskip}	

\[
	\prod_{r \in \mathbb{R}} \int_{\ln{1}}^{|r|} \exp(-iz) \sum_{k=1}^{100} \cos(kz) \, dz
\]
	
%%%%recalcar texto
\[
overbrace{\underbrace{(-1)^n}_{}}^{:)}
\]
%%%%%%%%%%%%%%%%%%%%%%%%%%%%%%_{} para escribir lo que es el corchete bajo
%%%%%%%%%%%%%%%%%%%%%%%%%%%%%%{}para escribir lo que es el corchete alto


\[
	Tx=0 \quad \text{si y sólo si} \quad x=0.
\]
	
\[
	\Bigg [ \dfrac{1}{2} \bigg \{ 4x \Big ( 8y \left( 7z \big ( 3x+4w( \dfrac{1}{4} + u) \Bigg ] \bigg \} \Big ) ) \right] \}
\]	
	
%%%%%%%%%%%%ALINEACIÓN

%%%%%%%%%%%% no mezclar \( \) [\ \] begin{align}

\begin{align}
\sum_{i=0}^{+ \infty} a^i  = 1 + a + a^2 + a^3 + \ldots	
\\
	 = \dfrac{1}{1+a}.
\end{align}

%%%%%%%%%%	
\begin{align}
\sum_{i=0}^{+ \infty} a^i & = 1 + a + a^2 + a^3 + \ldots & x & =y
\\
	& = \dfrac{1}{1+a}.							 & y & =z
\end{align}

%%%%%%%%%%%%

\begin{align}
\sum_{i=0}^{+ \infty} a^i & = 1 + a + a^2 + a^3 + \ldots & x & =y \nonumber
\\
	& = \dfrac{1}{1+a}.							 & y & =z \nonumber
\end{align}

%%%%%%%%%%%%%
\huge \Yinyang
%%%%%%%%%%%%%%
	
	\[
	\begin{Bmatrix}
	a & b & c \\
	d & e & f
	\end{Bmatrix}
\]	

	%%%%%%%%%%%%%
	%pmatrix bmatrix Bmatrix vmatrix Vmatrix para parentesis corchetes etc
	%small matrix para escribir matrices en texto
















	
	
\end{document}