% clase del documento
\documentclass[a4paper, 12pt]{article}
\usepackage[utf8]{inputenc}
\usepackage[spanish]{babel}
\usepackage[total={18cm, 21cm}, top=2cm, left=2cm]{geometry}
\usepackage{amsmath, amssymb, amsfonts, latexsym}
\usepackage{graphicx}
\usepackage{xcolor}

%comandos
\parindent = 10mm

\author{Antonio Merino \and Ivan Merino}

\title{Latex}

\date{\today}

%todo lo que esta atras, es el preambulo
%Contenido
\begin{document}
\maketitle


\section[título abreviado]{título completo}

\emph{Esto} de aquí ya sio se imprime wiiii. Esto de aquí ya sio se imprime wiiii.Esto de aquí ya sio se imprime wiiii.Esto de aquí ya sio se imprime wiiii.Esto de aquí ya sio se imprime wiiii.

\parskip =5mm

\emph{Algo de texto negro, \color{red} seguido por un fragmento rojo}, {\color {blue} finalmente algo de texto azul.}

\definecolor{db}{RGB}{100,50,2}
{\color{db}\emph{Esto} de aquí ya sio se imprime wiiii. Esto de aquí ya sio se imprime wiiii.Esto de aquí ya sio se imprime wiiii.Esto de aquí ya sio se imprime wiiii.Esto de aquí ya sio se imprime wiiii.}


\subsection[título abreviado]{título completo}

Esto de aquí ya sio se imprime wiiii. Esto de aquí ya sio se imprime wiiii.Esto de aquí ya sio se imprime wiiii.Esto de aquí ya sio se imprime wiiii.Esto de aquí ya sio se imprime wiiii.

Esto de aquí ya sio se imprime wiiii. Esto de aquí ya sio se imprime wiiii.Esto de aquí ya sio se imprime wiiii.Esto de aquí ya sio se imprime wiiii.Esto de aquí ya sio se imprime wiiii.

\begin{center}
	<<There are two types of people in this world, good and bad. The good sleep better, but the bad seem to enjoy the waking hours much more.>>
	\emph{Woody Allen}
\end{center}




\end{document}
Esto de aquí ya sio se imprime wiiii.