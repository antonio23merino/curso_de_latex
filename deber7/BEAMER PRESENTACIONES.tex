\documentclass[11pt,a4paper]{beamer}
\usetheme{Madrid} % temas en http://deic.uab.es/~iblanes/beamer_gallery/index_by_theme_and_color.html
\usepackage[T1]{fontenc}
\usepackage{latexsym}
\usepackage[utf8]{inputenc}
\usepackage[spanish,es-nolayout, es-nodecimaldot]{babel}
\usepackage{amsmath, amsfonts, amssymb}
\usepackage{bropd}
\usepackage{nicefrac}
\usepackage{stmaryrd}
\usepackage{listings}
\usepackage[nogin]{Sweave}
\usepackage{makeidx}
\usepackage{hyperref}
\usepackage{graphicx}
\DeclareGraphicsExtensions{.eps,.pdf,.png}

\setbeamertemplate{navigation symbols}{}

\usefonttheme{professionalfonts}
\setbeamercovered{transparent}

\newcommand{\N}{\mathbb{N}}
\newcommand{\R}{\mathbb{R}}
\newcommand{\llb}{\llbracket}
\newcommand{\rrb}{\rrbracket}

\DeclareMathOperator{\Cov}{Cov}
\DeclareMathOperator{\Var}{Var}
\DeclareMathOperator{\E}{E}

\allowdisplaybreaks
\setlength\parindent{0pt}
\renewcommand{\baselinestretch}{1.5}

\title{Simulación de \(\pi\) mediante variables \\ aleatorias uniformes}

\author{Andr\'{e}s Miniguano Trujillo}
\date{\today}


\date[\today] % (optional)
{Curso de \LaTeX, 201}
\subject{Estimador de pi}

\begin{document}

\begin{frame}  % Diapositiva de Titulo 
\titlepage
\end{frame}

\begin{frame}<beamer> % Diapositiva de Indice
\frametitle{Contenido}
\tableofcontents
\end{frame}

\AtBeginSection  % Diapositiva de inicio de seccion 
{
\begin{frame}
\begin{center}
\begin{beamercolorbox}[sep=8pt,center]{part title}
\usebeamerfont{part title}
\insertsection
\end{beamercolorbox}
\end{center}
\end{frame} 
}

\section{Proceso para estimar pi {}} 

\begin{frame}
\frametitle{Ley de los grandes números}
Mediante la ley de los grandes números se puede estimar el valor de \(\pi\) por medio de un algoritmo semejante a la regla de Laplace:
\end{frame}

\begin{frame}
\frametitle{Proceso}
    \begin{enumerate}[<+->]
        \item \texttt{Fijar un número natural \(n\) suficientemente grande e inicializar \(N=0\).}
        \item \texttt{Generar \(U_i \sim \mathcal{U}(0,1)\), vectores de dos componentes para cada \(i\in \llb 1,n \rrb \).}
        \item \texttt{Si \( U_{i,1}^2 + U_{i,2}^2 \leq 1 \), entonces \(N \leftarrow N+1\) .}
        \item \texttt{Calcular \( \tilde{\pi}_1 = 4\nicefrac{N}{n} \). Este valor es una aproximación de \(\pi\).}
    \end{enumerate}
\end{frame}

\begin{frame}
\frametitle{Proceso}
La idea es bastante sencilla: En el cuadrado de lado \(1\) y centrado en \( (\nicefrac 1 2, \nicefrac 1 2) \) generamos dos variables aleatorias uniformes (coordenadas de puntos). Luego contamos los puntos que caen dentro del círculo de radio \(1\) y centrado en \((0,0)\). Finalmente, una aproximación de \(\pi\) está dada por \(4\) veces la razón entre los puntos dentro del círculo y el número total de puntos simulados.
Pondré a prueba esta forma de simular \(\pi\):
\end{frame}

\begin{frame}
\frametitle{Forma para simular \(\pi\)}
\begin{eqnarray*}
> &n& <- 5000;\\
> &U& <- runif(2*n, 0, 1);\\
> &V& <- (U[1:n]^2 + U[(n+1):(2*n)]^2)\\
> &a& <- V <= 1\\
> &pi1& <- 4*sum(a)/n\\
\end{eqnarray*}

\end{frame}

\section{Aproximación de pi {}} 

\begin{frame}
\frametitle{Aproximación de \(\pi\)}
Obtengo la siguiente aproximación con su respectivo error:

\[
    \begin{array}{c|c|c}
        \pi_1 & \text{error} & \text{error relativo}
        \\ \hline
        3.1592 & 0.01761 & 0.0056
    \end{array}
\]
\end{frame}

 
\begin{frame}
\frametitle{Aproximación de \(\pi\)}
Ahora realizaré esta prueba varias veces:
\begin{eqnarray*}
&>& m <- 2000;      pi1 <- rep(0,m);\\
&>& for(k in 1:m)\{ \\
&+&     U <- runif(2*n, 0, 1);\\
&+&     V <- (U[1:n]^2 + U[(n+1):(2*n)]^2)\\
&+&     a <- V <= 1;    pi1[k] <- 4*sum(a)/n\\
&+& \} \\
\end{eqnarray*}
\end{frame}


\begin{frame}
\frametitle{Aproximación de \(\pi\)}
Presento a continuación el histograma de frecuencia relativa de los errores relativos:

\texttt{\textsl{> hist(abs(pi1-pi)/pi, density=100, border="beige", main="Histograma de error relativo", 
+      xlab="", ylab="Frecuencia relativa", freq=FALSE)}}

\end{frame}

\begin{frame}
\frametitle{Aproximación de \(\pi\)}
\begin{figure}[t]
\centering
\includegraphics<1>[scale=0.5, angle=0]{EPFL-histograma.pdf}
\end{figure}
\end{frame}
 
\end{document}